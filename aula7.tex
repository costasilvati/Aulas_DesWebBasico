\documentclass{beamer}
\usepackage[utf8]{inputenc}
\usepackage{url}
\usepackage{hyperref}
\graphicspath{{./fig/aula7}}

% Configurando layout para mostrar codigos C++
\usepackage{listings}
\lstset{
  language=HTML,
  basicstyle=\ttfamily\small, 
  keywordstyle=\color{blue}, 
  stringstyle=\color{red}, 
  commentstyle=\color{red}, 
  extendedchars=true, 
  showspaces=false, 
  showstringspaces=false, 
  numbers=left,
  numberstyle=\tiny,
  breaklines=true, 
  backgroundcolor=\color{green!10},
  breakautoindent=true, 
  captionpos=b,
  xleftmargin=0pt,
}


\title{Desenvolvimento Web Básico}
\subtitle{Aula 7}

\usetheme{lucid}

\begin{document}
\frame{
 \titlepage
}

%--------------------------------------------------------------------------
\begin{frame}{Na aula de hoje...} 
\tableofcontents 
\end{frame}
%--------------------------------------
\begin{frame}{Formulários}
  \begin{enumerate}
      \item Formulários são utilizados na maior parte das aplicações Web para coletar dados dos usuários e registrar alterações nos mesmos;
      \item Vamos rever os principais elementos de formulários, e como utiliza-los com JavaScript;
  \end{enumerate}
\end{frame}
%-----------------------------------------------------------------------
%-----------------------------------------------------------------
\section{Atividade de aula}
\begin{frame}{Atividade}{Envie seu repositório feito em aula}
  \begin{enumerate}
      \item Seguindo os passos da aula, livros indicados e, a leitura adicional recomendada: \textcolor{blue}{\href{https://docs.github.com/pt/pages/getting-started-with-github-pages/creating-a-github-pages-site}{Getting started with github pages}}, DESENVOLVA:
      \item Os passos realizados em aula;
      \item Envie no BlackBoard o link do repositório criado \textbf{em aula} e da GitHubPage;
  \end{enumerate}


\end{frame}
%-----------------------------------------------------------------------
\section{Leitura recomendada}
\begin{frame}{Leitura complementar}
 Para mais informações sobre Git e GitHub, leia:\\
  \vspace{0.6cm}
 \begin{columns}
   \begin{column}{0.4\textwidth}
%      \textbf{Capítulo 29}\\
 \cite{githubpages2022}\\
 E\\
 \cite{beer2015github}.
   \end{column}
   \begin{column}{0.4\textwidth}
   % \includegraphics[height=0.7\paperheight]{introgit.jpg} \\
   \end{column}

 \end{columns}
\end{frame}

%----------------------------------------------------------------------------
\section{Referências}

\begin{frame}{Referências}%[allowframebreaks]
\small
\begin{center}
\tiny
\bibliographystyle{apalike}
\bibliography{ref_aula}
\end{center}
\end{frame}

\end{document}

\end{document}