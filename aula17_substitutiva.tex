\documentclass{beamer}
\usepackage[utf8]{inputenc}
\usepackage{url}
\usepackage{hyperref}
\graphicspath{{./fig/aula7}}

% Configurando layout para mostrar codigos C++
\usepackage{listings}
\lstset{
  language=HTML,
  basicstyle=\ttfamily\small, 
  keywordstyle=\color{blue}, 
  stringstyle=\color{red}, 
  commentstyle=\color{red}, 
  extendedchars=true, 
  showspaces=false, 
  showstringspaces=false, 
  numbers=left,
  numberstyle=\tiny,
  breaklines=true, 
  backgroundcolor=\color{green!10},
  breakautoindent=true, 
  captionpos=b,
  xleftmargin=0pt,
}


\title{Desenvolvimento Web Básico}
\subtitle{Substitutiva}

\usetheme{lucid}

\begin{document}
\frame{
 \titlepage
}

%--------------------------------------------------------------------------
\begin{frame}{Na aula de hoje...} 
\tableofcontents 
\end{frame}
%--------------------------------------
\begin{frame}{Instruções}
Bem vindo a avaliação substitutiva! \\
\textbf{Orientações para a prova:}

  \begin{enumerate}
      \item Leia atentamente todas as questões e orientações;
      \item Desenvolva o que se pede por partes (isso ajuda a evitar acumulo de erros);
     \item Você pode utilizar todo o tempo da aula para desenvolver a sua avaliação, não tenha pressa!
     \item Cada item desenvolvido vale pontos (indicados na questão), faça tudo o que conseguir de cada item;
     \item Tudo será considerado, mesmo que incompleto; 
  \end{enumerate}
\end{frame}
%-----------------------------------------------------------------------
%-----------------------------------------------------------------
\section{Descrição}
\begin{frame}{Descrição}{Envie seu repositório feito em aula}
A avaliação substitutiva da disciplina de Desenvolvimento WEB Básico versará sobre todos os conetúdos do semestre (HTML, CSS, JS + GitHub)
  \begin{enumerate}
      \item Como serei avaliado?
      \item Questão prática:
      \item A prova terá uma questão que solicitará o desenvolvimento de HTML, CSS com Bootstrap e Javascript (JS);
      \item O aluno deve saber como implementar o código para Consumir uma API;
      \item Cadastrar, editar, consultar e listar, dados via JS usando JSON;
      \item Validações de formulários HTML;
  \end{enumerate}

\end{frame}
%-----------------------------------------------------------------------
\begin{frame}{Material permitido}
    \textbf{Material permitido:}\\

Cola institucional: 1 folha sulfite A4 manuscrita (escrita a mão), com tudo o que o aluno conseguir anotar (frente e verso).\\
\vspace{0,5cm}
Só pode realizar a avaliação os alunos que não atingiram 6,0 pontos de média.\\
\vspace{0,5cm}

A avaliação tem valor de 5,0 pontos, e substituirá a menor nota do aluno ( a substituição é automática via sistema no momento do lançamento da nota).\\
Bons estudos!
\\
\vspace{0,5cm}
Dúvidas devem ser enviadas para, juliana.silva@up.edu.br.
\end{frame}

%----------------------------------------------------------------------------
\begin{frame}{Pergunta}
\small
Desenvolva uma aplicação HTML, CSS e JS que receba informações de um aluno e apresente a resposta em uma nova página, conforme as orientações  abaixo. O formulário deve ter as seguintes características:
    \begin{itemize}
        \item (0,5 pontos) Um input para a digitação de nome que receba o nome digitado e salve em uma variável JS quando o usuário fizaer a submissão do formulário;
        \item (1,0 pontos) Quatro input para a digitação de nota1, nota2, nota3, nota4 que receba as notas digitadas e salve em um vetor JS  de 4 posições quando o usuário fizer a submissão do formulário;
        \item (1,0 pontos)  O formulário deve calcular a média do aluno e exibir o resultado automaticamente quando qualquer um dos campos de nota forem editados;
        \item (0,5 pontos) O formulário só deve seguir para outra página (submeter) se todas as notas estiverem preenchidas;
        
    \end{itemize}
    Continua..... 
\end{frame}
%-------------------------------------------------
\begin{frame}{Continuação...}
\small
\begin{itemize}
    \item (1,5 pontos) O formulário deve conter um botão "Registrar resultado", ao receber um clique, esse botão deve encaminhar o usuário para uma nova página que exiba:
        \item (0,5 pontos) "APROVADO" caso a média calculada seja maior que 6;
        \item (0,5 pontos) "EXAME" caso a média esteja entre 2 e 5,9;
        \item (0,5 pontos) "REPROVADO" caso a média seja menor que 2.
        \item (0,5 pontos) Faça o upload do código no Github e envie o link do repositório como resposta.
\end{itemize}
Caso não consiga publicar o repositório no Github, o aluno pode colocar os arquivos desenvolvidos em uma pasta na nuvem (Exemplo; Google Drive, One Drive, DropBox) e enviar o link de compartilhamento da mesma, links quebrados receberão nota 0. (Teste o link em uma guia anônima do navegador antes do envio!)
\end{frame}
%------------------------------------------------
\section{Referências}

\begin{frame}{Referências}%[allowframebreaks]
\small
\begin{center}
\tiny
\bibliographystyle{apalike}
\bibliography{ref_aula}
\end{center}
\end{frame}

\end{document}

\end{document}