\documentclass{beamer}
\usepackage[utf8]{inputenc}
\usepackage{url}
\usepackage{hyperref}
\graphicspath{{./fig/aula7}}

% Configurando layout para mostrar codigos C++
\usepackage{listings}
\lstset{
  language=HTML,
  basicstyle=\ttfamily\small, 
  keywordstyle=\color{blue}, 
  stringstyle=\color{red}, 
  commentstyle=\color{red}, 
  extendedchars=true, 
  showspaces=false, 
  showstringspaces=false, 
  numbers=left,
  numberstyle=\tiny,
  breaklines=true, 
  backgroundcolor=\color{green!10},
  breakautoindent=true, 
  captionpos=b,
  xleftmargin=0pt,
}


\title{Desenvolvimento Web Básico}
\subtitle{Aula 12 - Consumindo uma API}

\usetheme{lucid}

\begin{document}
\frame{
 \titlepage
}

%--------------------------------------------------------------------------
\begin{frame}{Na aula de hoje...} 
\tableofcontents 
\end{frame}
%--------------------------------------------------------------
\section{Referências}

\begin{frame}{Referências}%[allowframebreaks]
\small
\begin{center}
\tiny
\bibliographystyle{apalike}
\bibliography{ref_aula}
\end{center}
\end{frame}

\end{document}