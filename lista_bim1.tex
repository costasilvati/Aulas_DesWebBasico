(0,4 pontos) Seguindo os passos da aula, livros indicados e, a leituras recomendadas;
Desenvolva um formulário WEB de sua preferência que contenha os campos exibidos e monte um texto com as seleções do usuário.
Além disso o formulário deve mudar as cores com base em combinações de seleção diferentes;
Aplique CSS com Bootstrap para posicionamento e tipografia de elementos conforme a aula anterior
(0,4 pontos) Desenvolva um currículo pessoal (obrigat;oriamente do aluno) onde você descreve suas habilidades competências experiências e hobbies; Faça isso em formato de página WEB e estilize com Bootstrap;
(0,4 pontos) Desenvolva um sorteador de números aleatórios que realize o sroteio de um intervalo definido pelo usuário. Faça o sorteio em lingugem Javascript;
(0,4 pontos) Desenvolva um formulário WEB que receba 4 notas e o nome de um aluno:
Imprima a nota média do aluno;
Exiba também a situação do aluno -  aprovado (média > 6,0), reporvado (media < 2,0), recuperação (média entre 2,1 e 5,9);

Caso o aluno seja aprovado o fundo do texto de resposta deve ser verde;
Caso o aluno seja reprovado o fundo do texto deve ser vermelho;
Caso o aluno esteja de recuperação o funco do texto deve ser azul;
(0,4 pontos) Coloque o código desenvolvido em um repositório Github e envie o link do repositório como resposta, publique uma página Github como a página de exemplo: https://costasilvati.github.io/DesWebBasico/index.html , envie como resposta a página publicada, ela deve dar acesso a todas as páginas desenvolvidas nas atividades 1 a 4.
O repositório Github com o código das atividades é [repositório], a página github para visualização dos exercícios desenvolvidos é [paginas].